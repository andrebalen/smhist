\documentclass[a4paper]{report}
\usepackage[brazil]{babel}
\usepackage{graphicx}
\usepackage{imakeidx}
\usepackage{lettrine}
\usepackage[notlof]{tocbibind}

\makeindex[columns=3, title=keywords, intoc]

\title{ A esmeralda da ind\'ustria altomobil\'istica brasileira : Verde e Brutall}
\author{Andre Balen aka "el comodin" - smvb}
\date{agosto 2017}

\renewcommand*\contentsname{idx} 

\begin{document}
 
\maketitle
 
\tableofcontents

\clearpage

\addcontentsline{toc}{section}{Agradecimentos}
\section*{Agradecimentos}

Esta obra n\~ao seria poss\'ivel sem meus pais, Diva e Louren\c{c}o que me deram educa\c{c}\~ao, motiva\c{c}\~ao e as condi\c{c}\~oes
 para materializar este ensaio.

Agrade\c{c}o tambem a artista Lara Coletti pela ilustra\c{c}\~ao da capa e a Keista Lindo pela revis\~ao das vers\~oes em Portugu\^es e Ingl\^es.

Em especial um imenso obrigado a Ana Lidia Pimentel pelo seu lindo relato de momentos que viveu com seu pai e nos brinda com um conteudo maravilhoso
de uma conversa que tivemos.  



\clearpage

\addcontentsline{toc}{section}{Pref\'acio}
\section*{Pref\'acio}








O mundo n\~ao nos deve nada, ele j\'a estava aqui antes. \footnote{"Mark Twain"}






O universo n\~ao tem obriga\c{c}\~ao de fazer sentido pra voc\^e. \footnote{"Neil deGrasse Tyson"}






A natureza n\~ao \'e sutil, ela curte um alvoro\c{c}o! \footnote{em 18:44, 19 mar 2021 Andre Balen} 






\clearpage


\addcontentsline{toc}{section}{Janela Aberta, cabelos ao vento, o verso}
\section*{Janela Aberta, cabelos ao vento}

\lettrine[findent=2pt]{\fbox{\textbf{N}}}{ }em todas as pessoas no mundo tem a chance de um dia, materializar seus sonhos de crian\c{c}a,
 eu sou um destes sortudos, e ao longo destas p\'aginas vou lhes contar as formas com que o destino escolheu pra realizar este sonho... 

Esta \'e a saga\index{saga} de uma m\'aquina, que encontra seu motorista, e atrav\'es dele,  conta a historia de um autom\'ovel idealizado e
fabricado no brasil, por uma mulher \footnote{Ana Lidia Pimentel, com quem tive o prazer de conversar em meados de maio de 2021},
 filha de um empresario carioca, que administrou a f\'abrica de implementos agr\'icolas e ferrovi\'arios da fam\'ilia em Entre Rios.

% TODO: excluindo essa linha do arquivo final
% A saga brasilis da injambra e do espirito criativo\index{criativo} aliada a emo\c{c}\~ao

Diversas aventuras, que ao longo do tempo vivenciei, me deram a ideia de 
registrar as coisas mais interessantes, pois sei que daqui pra frente vai ser cada vez mais dif\'icil algu\'em vicenciar, por diversos motivos, a modernidade e complexidade gradual dos meios de transporte dispon\'iveis, a propria
natureza humana de querer sempre facilitar, abstrair e afastar-se ao m\'aximo possivel de saber como as coisas funcionam, da aventura do risco e do instinto\index{instinto}.

Pra mim, \'e isso que significa um autom\'ovel antigo, afastar um passo do materialismo, de querer sempre o carro do ano, da moda, da preocupa\c{c}\~ao natural de nossos dias
e aproximar do natural, do divertido, do instinto e de tudo que da gra\c{c}a a vida, a aventura, a emo\c{c}\~ao.

E n\~ao sou o \'unico que pensa assim, vide mujica, grande mestre do nosso tempo que nao me deixa mentir \dots

\'E disso que se trata esse livro, o prazer da vida contado por uma maquina de 6 cilindros\index{6 cilindros}, 4 rodas e um cora\c{c}\~ao.

Agrade\c{c}o a todos os amigos(as) que fizeram parte desta aventura, e desde j\'a dedico esta obra a todos voc\^es.

\clearpage

\addcontentsline{toc}{section}{descoberta}
\section*{A descoberta}
Nos anos 80, quando eu tinha meus 11 anos, havia uma santa matilde\index{santa matilde} branco perolada na minha cidade,
eu passava por ela algumas vezes, indo ou voltando da escola, sempre achei o carro mais bacana de todos\dots

\addcontentsline{toc}{figure}{o sm branco}
\begin{figure}[!htb]
\centering
\includegraphics{sm_bco_per}
\caption{SM branco Perolado}
\label{sm_bco}
\end{figure}

Esta id\'eia fortaleceu na minha mente, quando em meados dos 80 o filme "devolta para o futuro" chegou aos cinemas, naturamente como todos nessa \'epoca
fui assistir e como amante de tecnologia e fic\c{c}\~ao fiquei encantado pra n\~ao dizer, abestado por aquele delorean, naturalmente a imagem da SM perolada
voltou para minha cabe\c{c}a como sendo o mais proximo que eu podia achar no brasil de uma. Salvo \'e claro os detalhes t\'ecnicos da viagem no tempo q requer
a lataria exposta para conduzir os 1.21 GigaWats pela superficie permitindo o deslocamento temporal \dots

Certo dia, pela manh\~a, seu Louren\c{c}o, meu falecido pai, me pediu para ligar o carro, um ford scort ghia, para aquecer o motor.

Mesmo n\~ao sendo necess\'ario para um carro relativamente moderno para \'epoca, para mim foi uma experi\^encia \'unica,
at\'e aquele momento, eu nunca havia ligado um carro, pra mim foi um divisor de \'aguas para a vida adulta.

Obviamente eu n\~ao fazia ideia do funcionamento da embreagem e da marcha, liguei o carro com a marcha engatada e por sorte
nao demoli a frente do carro, foi a unica e ultima vez que tive a chance de fazer aquilo, mas a emo\c{c}\~ao do motor ligando ficou
na minha mem\'oria. 

Naquele mesmo ano vi mais algumas vezes a SM branco perolado e pensei que se algum dia eu pudesse ter um
autom\'ovel, seria uma SM, esse dia chegou em 2011.
\clearpage
Eu estava visitando uns amigos em rio grande e fui convidado para almo\c{c}ar \`a bordo com a tripula\c{c}\~ao do Atl\~antico Sul.

\addcontentsline{toc}{figure}{atl\^antico Sul}
\begin{figure}[!htb]
\centering
\includegraphics{atsul}
\caption{Navio oceanogr\'afico da FURG}
\label{at_sul}
\end{figure}

Durante o almo\c{c}o o assunto se dirigiu para autom\'oveis, lembrei do meu antigo sonho e compartilhei com o pessoal, nesta ocasi\~ao um dos marinheiros
me falou que havia uma santa matilde a venda em tapes, um conhecido.

Meu amigo Stefan, que havia convidado para o almo\c{c}o com sua equipe de trabalho, entusiamou-se com a informa\c{c}\~ao, e eu tambem, obviamente, e na semana 
seguinte fomos com mais um amigo de rio grande, Daniel Torres, vulgo "bala", outro entusiasta dos 6cilindros, ver a preciosidade.

No trajeto, rio grande - tapes, conversamos muito, sobre os motores, a divers\~ao, pescarias, mas o que nao saia da cabe\c{c}a, a SM.   
\clearpage

\addcontentsline{toc}{section}{primeira vinda}

\section*{primeira vinda}

Era verde, assim como seu interior, perfeita, parecia tudo ok quando vimos, exceto os pneus, esses estavam na capa da gaita

\addcontentsline{toc}{figure}{273a sm}
\begin{figure}[!htb]
\centering
\includegraphics{sm273}
\caption{273$^{a}$SM fonte: Cadastro Nacional da Santa Matilde}
\label{a 273a SM}
\end{figure}
\clearpage


\addcontentsline{toc}{section}{4marchas (v)arretado}
\section*{4marchas (v)arretado}

Ela veio com cambio manual 4 marchas, \'e meu primeiro carro, adorei. Havia um click em cada marcha,
mais tarde num posto no cassino meu camarada Daniel torres me mostrou por que, o cambio era exposto!
Uma serie de alavancas mexiam diretamente nas engrenagens, essas dimensionadas pra nao faltar as demais 
marchas, a 3a era extremamente poderosa, e a quarta parecia ser mais leve que uma sexta, andava muito. 
Mas nao consegui aproveitar o suficiente eu acredito, passei uns 2 anos assim e percebi que o fato de 
acavalar as marchas deixava o auto pouco confi\'avel e optei por botar uma caixa de 5 marchas mais 
moderna.
Certamente foi a coisa que eu mais tenho saudade, acredito que todos que tiveram esta oportunidade 
se lembram dessa caixa de 4 marchas varetada.

\addcontentsline{toc}{figure}{4 marchas opala varetada}
\begin{figure}[!htb]
\centering
\includegraphics{caixa4}
\caption{caixa 4 marchas com click. fonte: google}
\label{4 marchas opala varetada}
\end{figure}
\clearpage


\addcontentsline{toc}{section}{Amaral Ferrador}
\section*{Amaral Ferrador}

Esse foi um acampamento que a matilde mostrou a que veio, fui encontrar uns amigos acampados durante
um feriado de NS.Navegantes na beira do rio, um lugar maravilhoso e eu nunca tinha ido.
Milagrosamente minha companheira da \'epoca aceitou o desafio, pegamos a estrada no meio da tarde e 
na estrada comecou uma chuva muito forte.
Estrada de chao, chuva forte, local que eu n\~ao conhecia, uma receita pra imprevistos mas n\~ao,
a sm nos levou seguros, encontrei o acampamento sem muito problema e os dias que seguiram foram muito bons
No domingo eu estava no meio do rio que era raso, eu sentado a agua batia no peito, vi ao longe se aproximar
a prociss\~ao que era composta por alguns barcos sendo o primeiro, o que carrega a imagem da santa.
Sai de onde eu estava pra n\~ao atrapalhar o movimento e logo que passaram voltei pro meu local para acompanhar.

Existem lugares fantasticos no interior do estado, um carro como a sm \'e perfeito para explora-los.


\addcontentsline{toc}{section}{reflex\~oes sobre o impacto ambiental}
\section*{reflex\~oes sobre o impacto ambiental}

Desde que me conheco por gente essa preocupac\~ao permeia a minha existencia, sou filho de biologos, e como tal
sempre procuro fazer essa reflex\~ao.
Estamos na era dos automoveis el\'etricos, e sou engenheiro eletricista, nada mais natural do que imaginar isso,
converter a santa matilde para el\'etrico. N\~ao farei este sacril\'egio, mas nao posso deixar de pensar nesta possibilidade.
Aqui na minha cidade existe uma empresa de motores e geradores, a weg, e outra empresa de colegas meus da puc, a fueltec, que
desenvolve sistemas de injecao mais eficientes e esta no momento fazendo kits de controle de torque para motores eletricos da weg
para automoveis como o fusca e o gol.
Existem prototipos maravilhosos, mas o custo obviamente eh bastante elevado.
Sobre o impacto, o simples fato de eu nao ter trocado de carro e ter adquirido uma sm usada ja me da bastante credito, apesar de
nao ter a eficiencia de um carro moderno. Entao neste quesito tenho a consciencia limpa, mas estou disponivel para argumentacao,
esta conversa me interessa bastante.



\addcontentsline{toc}{section}{fan clube}
\section*{fan clube}

Ao longo do tempo em que utilizei a santa matilde surgiram f\~ans um exemplo \'e este pessoal da unissinos, \footnote{http://www.budanga.com.br/2012/06/o-pessoal-hoje-eu-tava-chegando-no.html} 
Nesta mesma \'epoca um acontecimento inusitado colaborou para a iniciativa de escrever as aventuras.

\addcontentsline{toc}{figure}{inicio do fan clube}
\begin{figure}[!htb]
\centering
\includegraphics{Foto0296}
\caption{pessoal da unissinos fonte: www.budanga.com.br}
\label{fan clube SL }
\end{figure}


\addcontentsline{toc}{figure}{fan clube tecnosinos}
\begin{figure}[!htb]
\centering
\includegraphics{Foto0295}
\caption{pessoal da unissinos fonte: www.budanga.com.br}
\label{Blog do pessoal da Tecnosinos }
\end{figure}
\clearpage

\addcontentsline{toc}{section}{o destino, eh o caminho, nocoes de proposito}


\section*{festivais}
	Muitos planos para os festivais, carros antigos lembram eventos mais tribais, e isso na verdade sao os festivais de hoje em dia, 
        estamos em outubro de 2019 proximo mes havera um festival no qual a verde e brutal conhece bem, ja trabalhou duro, caiu em valetas
        foi rebocada por fitas de e saiu mais feliz 

\section*{permacultura}
	Minha jornada com este naco de tecnologia formado por diversas ideias e solu\c{c}\~oes t\'ipicas de n\'os brasileiros\footnote{ a Snata matilde, feita de fibra de vidro, o mesmo material das pranchas de surf, tem mais de uma fun\c{c}\~ao, apesar de poluir de alguma forma segue um dos princ\'ipios da permacultura } me levou a conhecer a permacultura.
 	Me apaixonei, participei de diverosos mutir\~oes, reduzi o que pude todas as formas de pegada que eu pude deixando apenas plantas por onde passo, adoro plantar de tudo, de arvores a legumes,
        Conheci os principios da agricultura sintr\'opica e em dado momento pensei a rela\c{c}\~ao do ser humano com seus meios de transporte, passei a me concentrar muito em utilizar\footnote{e exercitar} meu corpo nos deslocamentos do dia a dia, mais saude menos combustivel f\'ossil, e passei a usar o autom\'ovel apenas para tarefas mais nobres como visitar amigos que moram longe, acampamentos e mutir\~oes.
	Esta escolha me trouxe sa\'ude e felicidade e recomendo a quem quiser se sentir mais leve. 


\section*{acampamentos}
	existe um vies filosofico no quesito acampamentos, este eh o proposito primordial no qual eu investi meu tempo para obter o dinheiro necessario para possuir uma maquina que me facilite esta atividade


\section*{pedra bacon}

 lembro q uma vez eu tava sentado dentro do rio viajando nas pedrinhas eu sentado a agua pela barriga, bem rasinho, mas no meio do rio
 achei uma pedra q parecia um cubinho de bacon  com as camadas umas de areia transparente outra mais fosca, bem um bacon fritinho
 levantei o olhar e vi uma aglomeraçao na beira do rio, um km adiante daonde eu tava
 ai eu fiquei bolado pensando qq ta rolando.  Dali a pouco olho pra tras e vem uns barcos pequenos, com um povo descendo o rio
 com a imagem da santa, era feriado de navegantes,  eu acampando,  a galera rezando hehehehe,  foi muito doida a sensacao

 
\addcontentsline{toc}{section}{nova forma de acelerar}
\section*{nova forma de acelerar}

Precisei criar esta alternativa na saida de um aniversario, estava em S\~ao Leopoldo, na casa de uma colega de trabalho, curtindo um bolo e papeando com os colegas
Tava divertido, mas na hora de ir embora percebi q estava sem acelerador, achei estranho e verifiquei no carburador que o cabo estava partido a 5cm dele.
Ai imaginei algo, havia um barbante de nylon no porta malas, se eu pudesse deixar uma fresta no capo eu poderia usar esse cord\~ao para acelerar pela janela, poupando uns 200 reais de guincho.
E foi o que eu fiz, no come\c{c}o foi meio estranho coordenar a embreagem, mas duas lombadas depois eu ja tinha pegado a manha.
Voltei tarde da noite de S\~ao Leopoldo para Porto Alegre acelerando por um barbante preso no carburador direto pela janela.
Ahhh o mais importante, para nao ficar presa a corda deixei um pano preso na fresta do cap\^o deixando espa\c{c}o suficiente para n\~ao trancar acelerado. 
acontecido em 1 de fevereiro de 2011 verificar ano

\addcontentsline{toc}{section}{TODO}

\section*{prot\'otipo e ideias soltas}
Uma coisa que eu notei \'e q me senti livre para criar pois sabendo como funciona eu facilmente posso adaptar algo e nao ficar na m\~ao


\section*{caseiro do sitio shambala}


Janeiro de 2018, meu grande amigo canadense sr. gersteimer ( vulgo dr. gonzo) me chamou pra tomar conta do seu sitio em viamao durante suas ferias de inverno, aproveitei pra levar a SM pra passear e fazer alguns pequenos reparos na porta do motorista enquanto curto um pouco de natureza, fogueiras, uma cachorrada marota e o novo amigo Valentin ( o cavalo da tes, filha deste meu amigo) foi um m\^es divertido com varias idas e vindas com a santinha, ela se comportou divinamente, sou muito feliz por poder rodar com esse carrinho. 

\addcontentsline{toc}{section}{Devotos da Matilde, a santa}
\section*{Devotos da Matilde, a santa}

Agosto de 2018, fazem meses que nao rodo com a santa matilde, puro desleixo mesmo, h\'a meses tenho combinado com o amigo Lee Ohn ( se eu for contar as historias com esse camarada da quase outro livro ) de subtrai-lo de um gerador que esta ocupando espa\c{c}o na casa dele, e finalmente chegou o dia, aproveitei uma folga no meio da tarde para evadir o transito que esta cada vez mais ca\'otico nesta porto alegre e fiz a miss\~ao, a matilde estava coberta de p\'o, estava parada desde janeiro, mas incrivelmente um xorinho de supra direto no carburador e ela ronronou sem muito trabalho, fizemos a mao de carregar o gerador, e meu amigo virou mais um devoto da santa, aproveitei e mostrei como fazer uma liga\c{c}\~ao direta, ta na hora de consertar isso, mas nao consigo, funciona t\~ao bem e \'e t\~ao legal que mesmo tendo comprado os componentes para modernizar eu nao me presto a finalizar o trabalho. 

\section*{o medico e o monstro}

Todo monstro precisa de um m\'edico, e isso nao eh diferente para a santa matilde, quando me convenci a realizar este sonho, um dos motivos foi pra
aprender mais sobre mec\^anica, sempre fui interessado no assunto e a sm sem d\'uvida tem contribuido muito para este aprendizado.
Depois de alguns anos rodando com a SM sem muitos problemas cometi um erro de misturar oleos de diferentes viscosidades e acabei danificando o motor quase chegando a santa vitoria do palmar, em resumo a poucos metros da policia federal o motor perdeu bastante oleo e eu parei pra ver, estava em servi\'co um amigo o kern que me indicou um mecanico local pra resolver e seguir viagem, o que foi feito, quando cheguei devolta da viagem resolvi revisar o carro com meu vizinho, o Brugmann, descobrimos alguns problemas e desde esse tempo tem sido a pessoa a qual recorro quando nao consigo resolver o problema por minha conta, e gracas a isso ainda consigo ter aventuras e material pra este ensaio.
Ainda sobre mecanica, li uma materia no SMclube\footnote{http://www.santamatilde.com.br} que na epoca em que estavam escolhendo a mecanica pro carro estavam entre a alfa romeo e o opala, no caso a alfa nao permitiu que a mecanica fosse usada,  essa decisao ajudou a tornar poss\'ivel  utilizar a sm da forma que utilizo, pois seria bem mais dif\'icil e caro conseguir pe\c{c}as de alfa


\addcontentsline{toc}{section}{A criadora e a criatura}
\section*{A criadora e a criatura}

Recentemente tive o prazer de entrar em contato com a Engenheira Ana Lidia Pimentel que foi a pessoa que desenhou a SM, nesta conversa abordamos alguns
assuntos muito interessantes, como a origem do nome Santa matilde \dots

\dots Sim, o carro se chama SM por causa do nome da f\'abrica, Companhia Industrial Santa Matilde. A companhia foi fundada em 1916, pelo meu bisav\^o \footnote{ em conversa de 17 de maio de 2021 com eng. Ana Lidia Pimentel }
O nome, Santa Matilde, veio quando meu bisav\^o comprou uma mina de mangan\^es, de um senhor cuja esposa se chamava Matilde.
A princ\'ipio era minera\c{c}\~ao de mangan\^es, passou para o conserto de vagonetas e depois para a fabrica\c{c}\~ao de vag\~oes.
Tudo come\c{c}ou com o setor ferrovi\'ario, depois veio o setor de estruturas met\'alicas ( torres e defensas rodovi\'arias) e depois a parte agr\'icola. No in\'icio as grades e forrageiras e depois as colheitadeiras e tratores. 

Perguntei a ela sobre o processo de galvaniza\c{c}\~ao, se tinha liga\c{c}\~ao com o mangan\^es. 

\dots N\~ao, o zinco \'e a base da galvaniza\c{c}\~ao. no chassis e nas estruturas met\'alicas, era feita a galvaniza\c{c}\~ao a quente.

Outra curiosidade minha foi a respeito de como foi desenhar a SM \dots

\dots N\~ao foi uma tarefa. Meu pai sempre foi apaixonado por autom\'oveis, herdei isso dele. Ele tinha um Porsche 911 S Targa na \'Epoca. Era o carro de uso di\'ario dele. As pe\c{c}as come\c{c}aram a ficar caras e muito dif\'iceis de conseguir. Ele entrou na fila para comprar um Puma GTB. Estava demorando mais que o previsto. Foi ai que eu entrei na hist\'oria. Perguntei a ele, por que nao fazia um carro na f\'abrica? 
Tinhamos a galvaniza\c{c}\~ao e tamb\'em a fibra de vidro que j\'a era usada em algumas m\'aquinas agr\'icolas e vag\~oes de carga. Ele achou uma loucura a minha id\'eia.
Passou um tempo e um dia, enquanto eu estudava na prancheta, ele veio cheio de cat\'alogos e revistas, sentou-se ao meu lado e disse: Vamos fazer o carro.
Pegamos um pouco de cada carro de que gost\'avamos. Criamos um Monstro!\footnote{lendo isso eu me dei conta que o nome deste livro, verde e brutal, que eu tirei de um desenho brasileiro que eu adoro e chama irm\~ao do jorel, nao poderia ser mais apropriado} 
Parecia o Frankenstein! rsrsrsrsrs
Ent\~ao come\c{c}amos a dar forma ao carro.
Foi assim que desenhei o carro. \'E claro que, sem meu pai, eu n\~ao teria feito grande coisa. Eu era s\'o uma estudante de engenharia mec\^anica, que gostava de carros. Ele foi pe\c{c}a fundamental para que o carro viesse a ser feito. Sem o conhecimento dele, o projeto seria invi\'avel. \footnote{acho que na lista de pessoas que agradecem a este momento esta eu, o ilustre zagallo e toda a torcida do corinthians ( ana lidia, me diga qual eh teu time favorito q eu troco isso aqui hehehehe }



\addcontentsline{toc}{section}{Aquilo que nos move}
\section*{Aquilo que nos move}

Muitas s\~ao as motiva\c{c}\~oes do ser humano, lembro de quando crian\c{c}a acordar cedo no domingo pra ver as corridas do Ayrton Senna, hoje
num 1o de maio de 2024 fazem 30 anos que perdemos este heroi nacional. Inclusive ele usava um bon\'e do nacional, meu amigo marcelo usou o 
mesmo bon\'e por uma d\'ecada ap\'os essa dolorosa perda, atingiu a todos de uma forma que palavras n\~ao podem expressar. 
Em Turin na italia, o museu Nacional do autom\'ovel\footnote{https://www.facebook.com/reel/953535999659761} esta expondo itens de Ayrton Senna em celeba\c{c}\~ao a vida e ao anivers\'ario 
de morte deste exemplo e inspira\c{c}\~ao. Certamente, Senna colaborou com meu amor pelo automobilismo, e provavelmente estou neste relato motivado
pela paix\~ao que ele tinha pelos carros. Me lembro de uma historia de quando ele mudou pra honda, e depois de algumas voltas, em que ele
tinha levado a m\'aquina ao limite, parou nos boxes e relatou aos engenheiros que ele havia trincado uma biela e queria usar o reserva.
Os engenheiros n\~ao acreditaram que ele poderia saber disso apenas dirigindo, mas foram l\'a desmontaram o motor, fizeram diagn\'ostico de
imagem dos componentes e acharam a trica na pe\c{c}a exatamente como Ayrton havia relatado, um g\^enio, incontestavel. Muitos anos num mesmo carro
faz a gente come\c{c}ar a ter este tipo de sensibilidade, mes passado eu sabia que estava com velas sujas s\'o de ouvir o meu motor.


\clearpage

\addcontentsline{toc}{section}{Anexos}
\addcontentsline{toc}{section}{Propagandas da \'epoca de lan\c{c}amento}
\addcontentsline{toc}{section}{Esquema el\'etrico da santa matilde}
\addcontentsline{toc}{section}{Lista de equival\^encia de pe\c{c}as}


\printindex
 
\end{document}
